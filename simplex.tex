\documentclass{article}
\usepackage{colortbl} 
\begin{document}
\paragraph{Einführung}
Dies ist eine Erklärung des Simplex Algorithmus. Weshalb es ihn gibt und wie er funktioniert\\
Zu aller erst sollte geklärt werden wann der simplex überhaupt benutzt wird.\\\\ 
\paragraph{Wann wird der Simplex benutzt?}
Der Simplex wird benutzt um eine maximierungs Problem zu lösen
\paragraph{Welche Bedingungen müssen vorherrschen?}
\begin{itemize}
	\item{Das Problem muss in der \textbf{Normalform} vorliegen}
	\item{Die \textbf{rechte Seite} darf \textbf{nicht negativ} sein!!! Ganz wichtig Niko}
	\item{Es muss eine \textbf{Einheitsmatrix} unter den Schlupfvariabeln vorliegen}
\end{itemize}
\paragraph{Wie sieht die Normalform aus}
Zu aller erst unser Maximierungsproblem\\
$Max F(x) = 81x_{1} + 60x_{2}$\\\\
Dann unsere Nebenbedingungen auch Restriktionen genannt
\begin{enumerate}
	\item{$2x_{1} + 2x_{2} \leq 16 $}
	\item{$4x_{1} + 2x_{2} \leq 24 $}
	\item{$4x_{1} + 6x_{2} \leq 36 $}
	\item[]{$x_{1},x_{2} \geq 0$}
\end{enumerate}
Dadurch sieht die Normalform also so aus\\
\begin{enumerate}
	\item{$2x_{1} + 2x_{2} + x_{3} = 16 $}
	\item{$4x_{1} + 2x_{2} + x_{4} = 24 $}
	\item{$4x_{1} + 6x_{2} + x_{5} = 36 $}
	\item[]{$x_{1},x_{2} \geq 0$}
\end{enumerate}
Nun haben wir auch schon den ersten Schritt fertig wir haben eine Normalform
\paragraph{Nun legen wir ein Simplex Tableu an}\mbox{}\\
\begin{tabular}{|l|l|l|l|l|l|l|l|l|}
	\hline
	Zeile & BV & $x_{1}$&$x_{2}$&$x_{3}$&$x_{4}$&$x_{5}$&$b_{i}$&$b_{i}/a_{is}$\\
	\hline
	(1)&$x_{3}$&2&2&1&0&0&16& \\
	\hline
	(2)&$x_{4}$&4&2&0&1&0&24& \\
	\hline
	(3)&$x_{5}$&4&6&0&0&1&36& \\
	\hline
	(3')&F&-80&-60&0&0&0&0&\\
	\hline
\end{tabular}\\\\
Perfekt dies ist die Grundform von unserem Simplex Tableu
\paragraph{Nun suchen wir das sogenannte Pivot Element}
Dafür suhen wir in Zeile (3') nach dem höchsten negativen Wert der BVs also in diesem Fall 80\\\\
\begin{tabular}{|l|l|l|l|l|l|l|l|l|}
	\hline
	Zeile & BV & $x_{1}$&$x_{2}$&$x_{3}$&$x_{4}$&$x_{5}$&$b_{i}$&$b_{i}/a_{is}$\\
	\hline
	(1)&$x_{3}$&2&2&1&0&0&16& \\
	\hline
	(2)&$x_{4}$&4&2&0&1&0&24& \\
	\hline
	(3)&$x_{5}$&4&6&0&0&1&36& \\
	\hline
	(3')&F&\cellcolor{red}-80&-60&0&0&0&0&\\
	\hline
\end{tabular}\\\\
Es geht also um folgende Spalte (nun Rot eingefärbt) auch genannt Pivotspalte\\\\
\begin{tabular}{|l|l|l|l|l|l|l|l|l|}
	\hline
	Zeile & BV & $x_{1}$&$x_{2}$&$x_{3}$&$x_{4}$&$x_{5}$&$b_{i}$&$b_{i}/a_{is}$\\
	\hline
	(1)&$x_{3}$&\cellcolor{red}2&2&1&0&0&16& \\
	\hline
	(2)&$x_{4}$&\cellcolor{red}4&2&0&1&0&24& \\
	\hline
	(3)&$x_{5}$&\cellcolor{red}4&6&0&0&1&36& \\
	\hline
	(3')&F&\cellcolor{red}-80&-60&0&0&0&0&\\
	\hline
\end{tabular}\\\\
Nun teilen wir die Werte der Spalte $b_{i}$ durch die Werte der Rot markierten Spalte und tragen es in die Spalte $b_{i}/a_{is}$\\ 
\begin{tabular}{|l|l|l|l|l|l|l|l|l|}
	\hline
	Zeile & BV & $x_{1}$&$x_{2}$&$x_{3}$&$x_{4}$&$x_{5}$&$b_{i}$&$b_{i}/a_{is}$\\
	\hline
	(1)&$x_{3}$&\cellcolor{red}2&2&1&0&0&16&16/2 = 8\\
	\hline
	(2)&$x_{4}$&\cellcolor{red}4&2&0&1&0&24&24/4 = 6\\
	\hline
	(3)&$x_{5}$&\cellcolor{red}4&6&0&0&1&36&36/4 = 9\\
	\hline
	(3')&F&\cellcolor{red}-80&-60&0&0&0&0&\\
	\hline
\end{tabular}\\\\
Nun suchen wir aus dieser Spalte den kleinsten Wert heraus und markieren ihn Gelb\\
\begin{tabular}{|l|l|l|l|l|l|l|l|l|}
	\hline
	Zeile & BV & $x_{1}$&$x_{2}$&$x_{3}$&$x_{4}$&$x_{5}$&$b_{i}$&$b_{i}/a_{is}$\\
	\hline
	(1)&$x_{3}$&\cellcolor{red}2&2&1&0&0&16&16/2 = 8\\
	\hline
	(2)&$x_{4}$&\cellcolor{red}4&2&0&1&0&24&\cellcolor{yellow}24/4 = 6\\
	\hline
	(3)&$x_{5}$&\cellcolor{red}4&6&0&0&1&36&36/4 = 9\\
	\hline
	(3')&F&\cellcolor{red}-80&-60&0&0&0&0&\\
	\hline
\end{tabular}\\\\
Wir haben nun also unsere Pivotzeile nämlich die Gelb eingefärbte\\
\begin{tabular}{|l|l|l|l|l|l|l|l|l|}
	\hline
	Zeile & BV & $x_{1}$&$x_{2}$&$x_{3}$&$x_{4}$&$x_{5}$&$b_{i}$&$b_{i}/a_{is}$\\
	\hline
	(1)&$x_{3}$&\cellcolor{red}2&2&1&0&0&16&16/2 = 8\\
	\hline
	(2)&\cellcolor{yellow}$x_{4}$&\cellcolor{yellow}4&\cellcolor{yellow}2&\cellcolor{yellow}0&\cellcolor{yellow}1&\cellcolor{yellow}0&\cellcolor{yellow}24&\cellcolor{yellow}24/4 = 6\\
	\hline
	(3)&$x_{5}$&\cellcolor{red}4&6&0&0&1&36&36/4 = 9\\
	\hline
	(3')&F&\cellcolor{red}-80&-60&0&0&0&0&\\
	\hline
\end{tabular}\\\\
An dem Punkt wo sich Pivotzeile und Spalte treffen(Grün eingefärbt) ist unser Pivot Element\\
\begin{tabular}{|l|l|l|l|l|l|l|l|l|}
	\hline
	Zeile & BV & $x_{1}$&$x_{2}$&$x_{3}$&$x_{4}$&$x_{5}$&$b_{i}$&$b_{i}/a_{is}$\\
	\hline
	(1)&$x_{3}$&\cellcolor{red}2&2&1&0&0&16&16/2 = 8\\
	\hline
	(2)&\cellcolor{yellow}$x_{4}$&\cellcolor{green}4&\cellcolor{yellow}2&\cellcolor{yellow}0&\cellcolor{yellow}1&\cellcolor{yellow}0&\cellcolor{yellow}24&\cellcolor{yellow}24/4 = 6\\
	\hline
	(3)&$x_{5}$&\cellcolor{red}4&6&0&0&1&36&36/4 = 9\\
	\hline
	(3')&F&\cellcolor{red}-80&-60&0&0&0&0&\\
	\hline
\end{tabular}\\\\
\paragraph{Soweit so gut, Nun fängt der Austauschschritt an}
wir haben unser Tableu mit dem Pivot Element\\\\
\begin{tabular}{|l|l|l|l|l|l|l|l|l|}
	\hline
	Zeile & BV & $x_{1}$&$x_{2}$&$x_{3}$&$x_{4}$&$x_{5}$&$b_{i}$&$b_{i}/a_{is}$\\
	\hline
	(1)&$x_{3}$&2&2&1&0&0&16&16/2 = 8\\
	\hline
	(2)&$x_{4}$&\cellcolor{green}4&2&0&1&0&24&24/4 = 6\\
	\hline
	(3)&$x_{5}$&4&6&0&0&1&36&36/4 = 9\\
	\hline
	(3')&F&-80&-60&0&0&0&0&\\
	\hline
\end{tabular}\\\\
Nun legen wir ein neues Tableu an, mit dem ausgetauschten $x_{4}$ und $x_{1}$ und neuen Zeilennummern\\
\begin{tabular}{|l|l|l|l|l|l|l|l|l|}
	\hline
	Zeile & BV &\cellcolor{blue} $x_{1}$&$x_{2}$&$x_{3}$&$x_{4}$&$x_{5}$&$b_{i}$&$b_{i}/a_{is}$\\
	\hline
	(4)&$x_{3}$&&&&&&&\\
	\hline
	(5)&\cellcolor{blue}$x_{1}$&&&&&&&\\
	\hline
	(6)&$x_{5}$&&&&&&&\\
	\hline
	(6')&F&&&&&&&\\
	\hline
\end{tabular}\\\\
Nun wird das Pivotelement durch Teilen auf 1 gebracht und der Rest der Zeile wird durch den gleichen Betrag geteilt. Hier also 4. Wir färben es grün ein\\ 
\begin{tabular}{|l|l|l|l|l|l|l|l|l|l|}
	\hline
	Zeile & BV &$x_{1}$&$x_{2}$&$x_{3}$&$x_{4}$&$x_{5}$&$b_{i}$&$b_{i}/a_{is}$&Operation\\
	\hline
	(4)&$x_{3}$&&&&&&&&\\
	\hline
	(5)&$x_{1}$&\cellcolor{green}1&\cellcolor{green}$\frac{1}{2}$&\cellcolor{green}0&\cellcolor{green}$\frac{1}{4}$&\cellcolor{green}0&\cellcolor{green}6&&(2):4\\
	\hline
	(6)&$x_{5}$&&&&&&&&\\
	\hline
	(6')&F&&&&&&&&\\
	\hline
\end{tabular}\\\\
Nun muss für die selbe Spalte wie das Pivot Element 0 eingetragen werden bzw. wird die Zeile so oft auf die anderen Addiert bzw. Subtrahiert bis es 0 ergibt. Dazu gleich mehr(wir haben es Gelb eingefärbt)\\
\begin{tabular}{|l|l|l|l|l|l|l|l|l|l|}
	\hline
	Zeile & BV &$x_{1}$&$x_{2}$&$x_{3}$&$x_{4}$&$x_{5}$&$b_{i}$&$b_{i}/a_{is}$&Operation\\
	\hline
	(4)&$x_{3}$&\cellcolor{yellow}0&&&&&&&\\
	\hline
	(5)&$x_{1}$&1&$\frac{1}{2}$&0&$\frac{1}{4}$&0&6&&(2):4\\
	\hline
	(6)&$x_{5}$&\cellcolor{yellow}0&&&&&&&\\
	\hline
	(6')&F&&&&&&&&\\
	\hline
\end{tabular}\\\\
Nun für also für Zeile (4) bspw. muss Zeile(5) zweimal bzw. Zeile (2) ein halbes mal abezogen werden(wir färben es grün ein)\\
\begin{tabular}{|l|l|l|l|l|l|l|l|l|l|}
	\hline
	Zeile & BV &$x_{1}$&$x_{2}$&$x_{3}$&$x_{4}$&$x_{5}$&$b_{i}$&$b_{i}/a_{is}$&Operation\\
	\hline
	(4)&$x_{3}$&\cellcolor{green}0&\cellcolor{green}1&\cellcolor{green}1&\cellcolor{green}$-\frac{1}{2}$&\cellcolor{green}0&\cellcolor{green}4&&(1)-2(5)\\
	\hline
	(5)&$x_{1}$&1&$\frac{1}{2}$&0&$\frac{1}{4}$&0&6&&(2):4\\
	\hline
	(6)&$x_{5}$&0&&&&&&&\\
	\hline
	(6')&F&&&&&&&&\\
	\hline
\end{tabular}\\\\
Das geliche für Zeile (6) (wieder Grün eingefärbt)\\
\begin{tabular}{|l|l|l|l|l|l|l|l|l|l|}
	\hline
	Zeile & BV &$x_{1}$&$x_{2}$&$x_{3}$&$x_{4}$&$x_{5}$&$b_{i}$&$b_{i}/a_{is}$&Operation\\
	\hline
	(4)&$x_{3}$&0&1&1&$-\frac{1}{2}$&0&4&&(1)-2(5)\\
	\hline
	(5)&$x_{1}$&1&$\frac{1}{2}$&0&$\frac{1}{4}$&0&6&&(2):4\\
	\hline
	(6)&$x_{5}$&\cellcolor{green}0&\cellcolor{green}4&\cellcolor{green}0&\cellcolor{green}-1&\cellcolor{green}1&\cellcolor{green}12&&(3)-4(5)\\
	\hline
	(6')&F&&&&&&&&\\
	\hline
\end{tabular}\\\\
Zu guter letzt natürlich noch die (6`) Zeile (hier Blau gefärbt)\\
\begin{tabular}{|l|l|l|l|l|l|l|l|l|l|}
	\hline
	Zeile & BV &$x_{1}$&$x_{2}$&$x_{3}$&$x_{4}$&$x_{5}$&$b_{i}$&$b_{i}/a_{is}$&Operation\\
	\hline
	(4)&$x_{3}$&0&1&1&$-\frac{1}{2}$&0&4&&(1)-2(5)\\
	\hline
	(5)&$x_{1}$&1&$\frac{1}{2}$&0&$\frac{1}{4}$&0&6&&(2):4\\
	\hline
	(6)&$x_{5}$&0&4&0&-1&1&12&&(3)-4(5)\\
	\hline
	(6')&F&\cellcolor{blue}0&\cellcolor{blue}-20&\cellcolor{blue}0&\cellcolor{blue}20&\cellcolor{blue}0&\cellcolor{blue}480&&(3')+20(5)\\
	\hline
\end{tabular}\\\\
Wären nun in der (6') Zeile keine negativen Zahlen hätten wir eine optimale Lösung. Da aber noch die -20 drin stehen können wir weiter optimieren. Hätten wir bereits eine optimale Lösung müssten wir nu noch die Spalte $b_{i}$ ablesen. Da dort die Werte zu den getauschten BVs stehen und der gesamt Wert.\\\\
Da dem aber nicht so ist müssen wir eine weitere Iteration machen und wieder schauen.\\
Dies habe ich hier gemacht und wie sie sehen, gibt es keine negative Zahl mehr in zeile (9'). Dies bedeutet das es nichts mehr zu optimieren gibt. Wir können nun die optimale Lösung ablesen. Ich habe es hierfür Grün eingefärbt. 
\begin{tabular}{|l|l|l|l|l|l|l|l|l|}
	\hline
	Zeile & BV &$x_{1}$&$x_{2}$&$x_{3}$&$x_{4}$&$x_{5}$&$b_{i}$&$b_{i}/a_{is}$\\
	\hline
	(7)&$x_{3}$&0&0&1&$-\frac{1}{4}$&-$\frac{1}{4}$&1&\\
	\hline
	(8)&\cellcolor{green}$x_{1}$&0&0&0&$\frac{3}{8}$&-$\frac{1}{8}$&\cellcolor{green}$\frac{9}{2}$&\\
	\hline
	(9)&\cellcolor{green}$x_{2}$&0&1&0&-$\frac{1}{4}$&$\frac{1}{4}$&\cellcolor{green}3&\\
	\hline
	(9')&\cellcolor{green}F&0&0&0&15&5&\cellcolor{green}540&\\
	\hline
\end{tabular}\\\\
\end{document}
