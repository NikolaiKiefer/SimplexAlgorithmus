\documentclass{article}
\usepackage{colortbl} 
\begin{document}
\paragraph{Einführung}
Dies ist eine Erklärung des Simplex Algorithmus. Weshalb es ihn gibt und wie er funktioniert\\
Zu aller erst sollte geklärt werden wann der simplex überhaupt benutzt wird.\\\\ 
\paragraph{Wann wird der Simplex benutzt?}
Der Simplex wird benutzt um eine maximierungs Problem zu lösen
\paragraph{Welche Bedingungen müssen vorherrschen?}
\begin{itemize}
	\item{Das Problem muss in der \textbf{Normalform} vorliegen}
	\item{Die \textbf{rechte Seite} darf \textbf{nicht negativ} sein!!! Ganz wichtig Niko}
	\item{Es muss eine \textbf{Einheitsmatrix} unter den Schlupfvariabeln vorliegen}
\end{itemize}
\paragraph{Wie sieht die Normalform aus}
Zu aller erst unser Maximierungsproblem\\
$Max F(x) = 81x_{1} + 60x_{2}$\\\\
Dann unsere Nebenbedingungen auch Restriktionen genannt
\begin{enumerate}
	\item{$2x_{1} + 2x_{2} \leq 16 $}
	\item{$4x_{1} + 2x_{2} \leq 24 $}
	\item{$4x_{1} + 6x_{2} \leq 36 $}
	\item[]{$x_{1},x_{2} \geq 0$}
\end{enumerate}
Dadurch sieht die Normalform also so aus\\
\begin{enumerate}
	\item{$2x_{1} + 2x_{2} + x_{3} = 16 $}
	\item{$4x_{1} + 2x_{2} + x_{4} = 24 $}
	\item{$4x_{1} + 6x_{2} + x_{5} = 36 $}
	\item[]{$x_{1},x_{2} \geq 0$}
\end{enumerate}
Nun haben wir auch schon den ersten Schritt fertig wir haben eine Normalform
\paragraph{Nun legen wir ein Simplex Tableu an}\mbox{}\\
\begin{tabular}{|l|l|l|l|l|l|l|l|l|}
	\hline
	Zeile & BV & $x_{1}$&$x_{2}$&$x_{3}$&$x_{4}$&$x_{5}$&$b_{i}$&$b_{i}/a_{is}$\\
	\hline
	(1)&$x_{3}$&2&2&1&0&0&16& \\
	\hline
	(2)&$x_{4}$&4&2&0&1&0&24& \\
	\hline
	(3)&$x_{5}$&4&6&0&0&1&36& \\
	\hline
	(3')&F&-80&-60&0&0&0&0&\\
	\hline
\end{tabular}\\\\
Perfekt dies ist die Grundform von unserem Simplex Tableu
\paragraph{Nun suchen wir das sogenannte Pivot Element}
Dafür suhen wir in Zeile (3') nach dem höchsten negativen Wert der BVs also in diesem Fall 80\\\\
\begin{tabular}{|l|l|l|l|l|l|l|l|l|}
	\hline
	Zeile & BV & $x_{1}$&$x_{2}$&$x_{3}$&$x_{4}$&$x_{5}$&$b_{i}$&$b_{i}/a_{is}$\\
	\hline
	(1)&$x_{3}$&2&2&1&0&0&16& \\
	\hline
	(2)&$x_{4}$&4&2&0&1&0&24& \\
	\hline
	(3)&$x_{5}$&4&6&0&0&1&36& \\
	\hline
	(3')&F&\cellcolor{red}-80&-60&0&0&0&0&\\
	\hline
\end{tabular}\\\\
Es geht also um folgende Spalte (nun Rot eingefärbt) auch genannt Pivotspalte\\\\
\begin{tabular}{|l|l|l|l|l|l|l|l|l|}
	\hline
	Zeile & BV & $x_{1}$&$x_{2}$&$x_{3}$&$x_{4}$&$x_{5}$&$b_{i}$&$b_{i}/a_{is}$\\
	\hline
	(1)&$x_{3}$&\cellcolor{red}2&2&1&0&0&16& \\
	\hline
	(2)&$x_{4}$&\cellcolor{red}4&2&0&1&0&24& \\
	\hline
	(3)&$x_{5}$&\cellcolor{red}4&6&0&0&1&36& \\
	\hline
	(3')&F&\cellcolor{red}-80&-60&0&0&0&0&\\
	\hline
\end{tabular}\\\\
Nun teilen wir die Werte der Spalte $b_{i}$ durch die Werte der Rot markierten Spalte und tragen es in die Spalte $b_{i}/a_{is}$\\ 
\begin{tabular}{|l|l|l|l|l|l|l|l|l|}
	\hline
	Zeile & BV & $x_{1}$&$x_{2}$&$x_{3}$&$x_{4}$&$x_{5}$&$b_{i}$&$b_{i}/a_{is}$\\
	\hline
	(1)&$x_{3}$&\cellcolor{red}2&2&1&0&0&16&16/2 = 8\\
	\hline
	(2)&$x_{4}$&\cellcolor{red}4&2&0&1&0&24&24/4 = 6\\
	\hline
	(3)&$x_{5}$&\cellcolor{red}4&6&0&0&1&36&36/4 = 9\\
	\hline
	(3')&F&\cellcolor{red}-80&-60&0&0&0&0&\\
	\hline
\end{tabular}\\\\
Nun suchen wir aus dieser Spalte den kleinsten Wert heraus und markieren ihn Gelb\\
\begin{tabular}{|l|l|l|l|l|l|l|l|l|}
	\hline
	Zeile & BV & $x_{1}$&$x_{2}$&$x_{3}$&$x_{4}$&$x_{5}$&$b_{i}$&$b_{i}/a_{is}$\\
	\hline
	(1)&$x_{3}$&\cellcolor{red}2&2&1&0&0&16&16/2 = 8\\
	\hline
	(2)&$x_{4}$&\cellcolor{red}4&2&0&1&0&24&\cellcolor{yellow}24/4 = 6\\
	\hline
	(3)&$x_{5}$&\cellcolor{red}4&6&0&0&1&36&36/4 = 9\\
	\hline
	(3')&F&\cellcolor{red}-80&-60&0&0&0&0&\\
	\hline
\end{tabular}\\\\
Wir haben nun also unsere Pivotzeile nämlich die Gelb eingefärbte\\
\begin{tabular}{|l|l|l|l|l|l|l|l|l|}
	\hline
	Zeile & BV & $x_{1}$&$x_{2}$&$x_{3}$&$x_{4}$&$x_{5}$&$b_{i}$&$b_{i}/a_{is}$\\
	\hline
	(1)&$x_{3}$&\cellcolor{red}2&2&1&0&0&16&16/2 = 8\\
	\hline
	(2)&\cellcolor{yellow}$x_{4}$&\cellcolor{yellow}4&\cellcolor{yellow}2&\cellcolor{yellow}0&\cellcolor{yellow}1&\cellcolor{yellow}0&\cellcolor{yellow}24&\cellcolor{yellow}24/4 = 6\\
	\hline
	(3)&$x_{5}$&\cellcolor{red}4&6&0&0&1&36&36/4 = 9\\
	\hline
	(3')&F&\cellcolor{red}-80&-60&0&0&0&0&\\
	\hline
\end{tabular}\\\\
An dem Punkt wo sich Pivotzeile und Spalte treffen(Grün eingefärbt) ist unser Pivot Element\\
\begin{tabular}{|l|l|l|l|l|l|l|l|l|}
	\hline
	Zeile & BV & $x_{1}$&$x_{2}$&$x_{3}$&$x_{4}$&$x_{5}$&$b_{i}$&$b_{i}/a_{is}$\\
	\hline
	(1)&$x_{3}$&\cellcolor{red}2&2&1&0&0&16&16/2 = 8\\
	\hline
	(2)&\cellcolor{yellow}$x_{4}$&\cellcolor{green}4&\cellcolor{yellow}2&\cellcolor{yellow}0&\cellcolor{yellow}1&\cellcolor{yellow}0&\cellcolor{yellow}24&\cellcolor{yellow}24/4 = 6\\
	\hline
	(3)&$x_{5}$&\cellcolor{red}4&6&0&0&1&36&36/4 = 9\\
	\hline
	(3')&F&\cellcolor{red}-80&-60&0&0&0&0&\\
	\hline
\end{tabular}\\\\
\end{document}
